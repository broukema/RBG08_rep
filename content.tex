

\abstract{\articleABSTRACT} %% seems to be missing in the template
\supercite{RBG08}

\section{Introduction}

This paper studies the reproducibility of the main observational
results of a cosmic topology research paper published by myself and
co-authors in 2008\supercite{RBG08}. The paper used the
surface-of-last-scattering optimal cross-correlation method of finding
a preferred orientation of the fundamental domain of the spatial
section of the Universe, under the working hypothesis that the spatial
section is a Poincaré dodecahedral space\supercite{LumNat03}. The
code was developed by me, with comments provided by my coauthors. The
results that should be reproduced are those that use the method
described in Section 3.2 of RBG08, and the observational analysis
results described in Section 4.2, displayed in Figs.~3, 4, 5 and given
numerically in Tables 2 and 3 of RBG08. Related cosmic topology papers
by other authors are published with no references to software package
details or software licences.

The reason for attempting and documenting the reproducibility of this
paper is that not only are many papers in astronomy\supercite{Allen18}
and other fields still published without providing the full
empirical data sets and source code under free-software licences,
but even those that provide free-licensed software and input data
may be difficult to reproduce\supercite{Ioannidis2009,Chang15,Stodden18}.
While observational data in cosmology are usually made available online
with high-quality documentation, often after an embargo period,
free-licensed software in the field of cosmic topology,
in particular, library functions
for defining matched circles in the cosmic microwave background
or matched \emph{discs} in extragalactic 3-dimensional comoving space\supercite{RK11},
is only recorded in the scientific literature in papers published by my research group.

To document and help analyse the success and difficulties in reproducing
scientific results in this context,
the editors of {\it ReScience C} posed the ``Ten Years Reproducibility
Challenge'', a request that scientists attempt to
reproduce the main results of \emph{their own}
peer-reviewed scientific research papers that had been
published before 1 January 2010, and document the method and results
in {\it ReScience C}\supercite{TenYrChallenge20}.

\section{Method}

\subsection{}
The first steps for trying to reproduce the original results
of RBG08 were to (re-)read the appropriate sections of the paper,
initially taking the view of a non-author.
\begin{enumerate}
\item
  Section 2.1\supercite{RBG08} states that the analysis method of
  Section 3.2 requires the three files at URLs listed in footnotes 1,
  2, 3 on the same page. These files represent two versions of an
  all-sky map of the Universe mostly representing cosmic microwave
  background emission at 10{\hGpc} (comoving) from the Earth as
  observed by the Wilkinson Microwave Anisotropy Probe
  (WMAP)\supercite{WMAP5Hinshaw}, and the ``kp2'' mask to enable
  analysis that avoids the most contaminated regions of the sky. These
  files need to be downloaded.
\item
  Footnote 7\supercite{RBG08} indicates that {\sc circles-0.3.2.1}, to
  be found at the URL \url{http://cosmo.torun.pl/GPLdownload/dodec/},
  provides the software for generating the figures and tables. This
  software needs to downloaded from
  \url{http://cosmo.torun.pl/GPLdownload/dodec/circles-0.3.2.1.tar.gz}.
\end{enumerate}

The next step was to develop a script on a {\sc git} repository server
that satisfies the requirements of the international scientific
community, specifically the International Science
Council\supercite{ISCFreedoms}, by not blocking access to scientists
of any countries or territories.  As of 2020, several popular {\sc
  git} repository servers block access to scientists and other
residents of several territories\supercite{Github2020}. A shift to
servers acceptable under international scientific ethical standards is
underway, but incomplete as of early 2020.

The remaining planned steps were to implement the minimal number of
updates to make the code work and replicate the original results,
using modern hardware and a modern software environment. Footnote 7 of
RBG08 warns that ``These [{\sc circles-0.3.2.1} and {\sc
    circles-0.3.8}] and earlier versions of the software require
medium to advanced {\sc GNU/Linux}, {\sc Fortran77} and {\sc C}
experience for a scientific user.'' There is no statement regarding
the particular compiler(s) used.  As far as I recall, it's very likely
that the widely used GNU fortran compiler of the time, {\sc g77}, was
used together with {\sc gcc}, as selected automatically by {\sc
  autotool} packages.

The system and hardware chosen for the reproduction project were an
{\sc AMD} computer running with a {\sc Debian GNU/Linux 9.12} system on an
{\sc x86\_64 Linux-4.9.0 kernel}.
The Fortran compiler chosen was {\sc GNU Fortran (Debian
  6.3.0-18+deb9u1) 6.3.0 20170516}.


\begin{table}
  \begin{tabular}{ll}
    \hline
    ded213c1c4cfdfe2ef92f7155b27d58c & wmap\_ilc\_5yr\_v3.fits \\
    fbc8b2518fdddf0a1e7b5acde99a748e & wiener5yr\_map.fits \\
    5aa3267dc6d69bf8c5f0a3a893e23960 & wmap\_kp2\_r9\_mask\_3yr\_v2.fits \\
    afbd67d8120c11e949eb0c414c2775f5 & circles-0.3.2.1.tar.gz \\
    \hline
  \end{tabular}
  \caption{Checksums (md5sums) of the data and the main software
    source code files of RBG08, use for the present reproducibility
    test.\protect\label{t-md5sums}}
\end{table}

\section{Results}

The overall script intended to carry out the full sequence of
downloads, configuring of packages, compiling of packages,
subdirectory user-level installation of packages, setting up of
calculation parameters, and running the main code, was set up as a
{\sc bash} script {\tt reproduce\_RBG08.sh}.

The full package aiming to reproduce the figures and tables listed
above is provided at \url{https://codeberg.org/boud/0807.4260},
named after the ArXiv identity of RBG08.

\subsection{Downloading data and software source code}

\begin{enumerate}
\item
  The URL in footnotes 1, 2 and 3\supercite{RBG08} gave clickable
  links that were split into two and not correctly clickable. The user
  needs to cut/paste the two halves of each URL in order to obtain the
  three data files. The data files were downloaded with no apparent
  problem, with md5sums as indicated in Table~\ref{t-md5sums}.
\item
  The file {\sc circles-0.3.2.1.tar.gz} with the md5sum indicated in
  Table~\ref{t-md5sums} was downloaded. It was included in
  the main git repository in its original form. Subsequent changes
  are recorded in the git history at
  \url{https://codeberg.org/boud/0807.4260}.
\end{enumerate}

\subsection{Compiling/debugging}

Fixes needed in order to successfully compile {\sc circles} include:
\begin{enumerate}
\item
  A Fortran 77 line that ended on one line with a $+$ symbol and
  started on the next line with another $+$ symbol (within the valid
  columns for standard Fortran 77) was apparently accepted by the {\sc
    gcc} family fortran compiler in 2008, but not now (2020). One of
  the $+$ symbols was removed.
\item
  A fitting algorithm {\sc gsl\_multifit\_covar} available in GNU
  Scientific Library ({\sc GSL}) versions 1.x was obsoleted; it is no
  longer present in modern 2.x versions of {\sc GSL}. With the aim of
  minimising the interventions required in the system, {\sc
    GSL-1.10} was downloaded and compiled from source, re-creating
  part of the original software environment.
\item
  Using the modern {\sc gfortran} compiler options {\tt -fcheck=bounds
    -Wall} to highlight likely sources of bugs due to insufficiently
  standard coding led to many warnings.  Checking of these warnings
  motivated many fixes that could be expected to either solve errors
  in running the main {\sc circles} package, or reduce the chance of
  calculational errors.
\item
  The autotools {\tt autoreconf} command was run in the main {\sc
    circles-0.3.2.1} directory and its subdirectories.
\item
  The {\sc cosmdist} package provided by default in a subdirectory of
  {\sc circles-0.3.2.1} was replaced by a
  download/configure/compile/install section of the main reproduction
  script, since {\sc cosmdist} is now available in an online
  git repository. The aim was to reduce the chance of {\sc cosmdist}
  being a blocking factor in reproduction of the calculations.
\item
  Memory allocation errors that occur for running {\sc circles}
  without previously definining environment variables for key
  values such as input filenames are present in {\sc
    circles-0.3.2.1}. These were most likely not noticed in RBG08
  because of the use of environment variables providing these
  values. Fixes to the front-end C file {\sc circles.c} were
  made with the intention of avoiding memory allocation errors,
  which are typically reported to the user as segmentation faults.
\end{enumerate}

\subsection{Dependencies}

The full list of dependencies listed -- not necessarily really used -- in
the final {\sc gfortran} compile operation that creates the {\sc circles}
binary executable is:
{\tt \mbox{\tt -lpgplot} -lpng lcosmdist -lisolat -lastromisc llapack -lcblas
--lf77blas -latlas lgfortran -lquadmath -lcfitsio -lcosmdist -lgsl
--lgslcblas -lm -lgcc -lX11 -lm}. \sloppy

\subsection{Running/debugging}

At the time of writing the original paper, the aim was that the use of GNU
tools to provide a free-licensed package configurable and compilable
with {\tt ./configure \&\& make} and a detailed {\tt ./circles
  -{}-help} command would be sufficient to enable easy reproduction by
a scientifically competent user. For example, invoking the {\sc
  circles} help option shows both single-hyphen, one-character options
and their equivalent double-hyphen, long options, such as
\mbox{{\tt -i,  -{}-cmb\_file\_raw=FILE cmb fits file of input data}}.
\sloppy

\fussy
My private notes of what were intended to record the most significant steps
taken in carrying out the project, along with more minor steps, were
used in the attempted reproduction of RBG08.  However, in trying to
reproduce these steps now, it is clear that the original notes were
not as complete and unambiguous as they should be.  For the purpose of
the current exercise in reproducibility, a completely fresh {\sc bash}
script was prepared, which verifies the sha512 checksums of input data
files and software packages that are downloaded from source rather
than provided within the security context of the host system. The
style of the new script is partly based on more recent attempts at
reproducibility in my own recent papers in which {\sc git} commit
hashes of the software\supercite{Roukema17silvir} and a {\sc bash}
script for running the full software\supercite{RO19flatness} were
provided, with some inspiration from the {\sc make}-based reproducibility
framework\supercite{Akhlaghi15} recently renamed as {\sc maneage}.

This situation illustrates the problem of ``insider knowledge'' being
required for the reproducibility of a paper, where ``insider'' also
includes knowledge that may still be coded in the scientist's brain,
but not in written form.

A more fundamental problem in terms of coding and software
environment evolution is that this code uses a C front end and
a Fortran 77 backend, configured and compiled together using {\sc autoconf}
tools, in a way that, to the best of my knowledge and that of
my co-authors, worked correctly in 2008.

The main files are {\tt circles\_f77.f}, 2023 lines long, named to
emphasise the expected obsolescence of the Fortran 77 language
standard; 21 Fortran 77 source files with a total of 11,616 lines of
code in {\tt lib/}; and 3415 lines of Fortran 77 code in the auxiliary
package {\tt astromisc/lib/}. For modernisation of this code, a
minimum approach would be to convert from Fortran 77 to Fortran 2008,
in which case Fortran--C compatibility conventions that are reasonably well
developed and implemented by the {\sc gcc/gfortran} family could be used.

For the purposes of this reproducibility test, the required re-coding
effort would be more than is presently justified. While this is, as
far as I know, the only free-licensed code for identified-circles
matching for the purposes of cosmic topology analysis for which
peer-reviewed research has been published, the techniques developed in
codes that are not publicly available under either free or non-free
licences have developed considerably since 2008.  Moreover, the
research field is extremely high risk: an observational confirmation
of the spatial topology of the Universe would be a historically
important discovery, but whether or not this measurement is feasible
remains highly speculative. To serve as a basis for long-term
projects in this particular field, the software would best be
rewritten according to modern standards of C and/or Fortran; and
the best tested, accurate, fast, well-coded free-licensed
auxiliary libraries, such as {\sc cgal} for geometrical purposes,
could be used to avoid ``reinventing the wheel''.

Thus, this reproducibility attempt was terminated with
an untraced bug that is very likely to be memory related. The behaviour
of the bug (or bugs) is inconsistent, yielding errors such as
\mbox{{\tt Fortran runtime error: Substring out of bounds:}}
\mbox{{\tt upper bound (7) of 'bin\_format' exceeds string length (4)}} or
\mbox{{\tt Fortran runtime error: Substring out of bounds:}}
\mbox{{\tt upper bound (2) of 'indir' exceeds string length (-1158212918)}}.
\sloppy

\section{Discussion}

Compiler evolution and Fortran/C compatibility are key elements of
difficulties in reproducing RBG08, mostly combined with some coding
that is not sufficiently robust. In 2008, {\sc g77} was an obvious
choice of Fortran compiler in the stable distribution of Debian
GNU/Linux. Since the Debian community already had at the time a solid
reputation in terms of software security, verification of licensing
and fully transparent and participatory decision-making, this seemed
like a wise choice for reproducibility.  The main developer of {\sc
  g77} had already announced his intention to stop maintaining the
project in 2001\supercite{Burley01g77}, but even by 2010, two years
after RBG08 was published, the community remained unclear regarding
the relationship between {\sc g95}, based on {\sc gcc}, versus
{\sc gfortran}, part of {\sc gcc}\supercite{Bosscher10g95}.


\section{Conclusion}

\fussy
It is ironical that in the field of cosmic topology, not only are most
software packages only available privately with unknown licences
(as is the case in astronomy as recently as 2015\supercite{Allen18}),
but the code that is explicitly public with peer-reviewed published
results has turned out to be less easy to reproduce than
expected. This is, unfortunately, consistent with typical reports on
science research paper reproducibility\supercite{Ioannidis2009,Chang15,Stodden18}.
The specific bottleneck
suspected of leading to memory errors in this case was that the effort
required to update the Fortran 77 files at the heart of the code,
interfaced with a C front end,
and compiled with the current {\sc gfortran} compiler from within {\sc gcc}
rather than with the older, discontinued\supercite{Burley01g77} {\sc g77}
compiler, risked being too great to be justified
on any short time scale. While Fortran has remained actively used by
scientists since more than half a century ago and in its modern
standards continue to be used actively, and the original {\sc
  circles} package was prepared using the powerful GNU {\sc
  autotools}, a robust interface and standards for compiling C and
Fortran code together have only evolved quite recently.

While the results of the paper are not as reproducible as they
appeared to be, the requirement of the Ten Years Challenge
for the code to be placed in an online git repository,
which in this case is \url{https://codeberg.org/boud/0807.4260},
resulted in confirmation that the source code is
fully free-licensed, including all libraries and other
auxiliary software packages, and the input data files
remain publicly available online.

\section*{Acknowledgments}
{\footnotesize Part of this research has been supported by
  the ``A next-generation worldwide quantum sensor network with optical atomic clocks'' project of the TEAM IV programme of the Foundation for Polish Science co-financed by the European Union under the European Regional Development Fund.
  Part of this research has been supported by
  the Polish MNiSW grant DIR/WK/2018/12.
  Part of this research has been supported by
  the Pozna\'n Supercomputing and Networking Center (PSNC) computational grant 314.}
