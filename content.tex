

\abstract{\articleABSTRACT} %% seems to be missing in the template


\section{Introduction}

The aim in \emph{this} paper is to reproduce the results that use the
method described in Section 3.2 and the observational
analysis results described in Section 4.2, displayed in Figs.~3, 4, 5
and given numerically in Tables 2 and 3, where the numbers here
indicate those in RBG08.

\section{Method}

The first steps planned for trying to reproduce the original results
were to read the appropriate sections of RBG08.
\begin{enumerate}
\item
  Section 2.1\supercite{RBG08} states that the analysis method of Section 3.2 requires
  the three files at URLs listed in footnotes 1, 2, 3 on the same page. These files
  represent two versions of an all-sky map of the Universe mostly representing cosmic microwave background
  emission at 10{\hGpc} (comoving) from the Earth as observed by
  the Wilkinson Microwave Anisotropy Probe (WMAP)\supercite{WMAP5Hinshaw}, and
  the ``kp2'' mask to enable analysis that avoids the most contaminated regions of
  the sky. These files are downloaded.
\item
  Footnote 7\supercite{RBG08} indicates that {\sc circles-0.3.2.1}, to be found at
  the URL \url{http://cosmo.torun.pl/GPLdownload/dodec/}, provides the software
  for generating the figures and tables. This software is downloaded as
  \url{http://cosmo.torun.pl/GPLdownload/dodec/circles-0.3.2.1.tar.gz}.
\end{enumerate}

The next step was to develop a script on a {\sc git} repository server
that satisfies the requirements of the international scientific community,
specifically the International Science Council\supercite{ISCFreedoms}, by
not blocking access to scientists of any countries or territories.
As of 2020, several popular {\sc git} repository servers block access to
scientists and other citizens of several territories\supercite{Github2020}.


\begin{table}
  \begin{tabular}{ll}
    \hline
    ded213c1c4cfdfe2ef92f7155b27d58c & wmap\_ilc\_5yr\_v3.fits \\
    fbc8b2518fdddf0a1e7b5acde99a748e & wiener5yr\_map.fits \\
    5aa3267dc6d69bf8c5f0a3a893e23960 & wmap\_kp2\_r9\_mask\_3yr\_v2.fits \\
    afbd67d8120c11e949eb0c414c2775f5 & circles-0.3.2.1.tar.gz \\
    \hline
  \end{tabular}
  \caption{Checksums (md5sums) of the data and software source code files
    of RBG08, calculated for the present paper.\protect\label{t-md5sums}}
\end{table}

\section{Results}

\subsection{Downloading data and software source code}

\begin{enumerate}
\item
  The URL in footnotes 1, 2 and 3\supercite{RBG08} gave clickable links that were split into
  two and not correctly clickable. The user needs to cut/paste the two halves of each URL
  in order to obtain the three data files. The data files were downloaded with no apparent
  problem, with md5sums as indicated in Table~\ref{t-md5sums}.
\item
  The file {\sc circles-0.3.2.1.tar.gz} with the md5sum indicated in Table~\ref{t-md5sums} was
  downloaded.
\end{enumerate}

\subsection{Compiling}

Fixes needed in order to successfully compile {\sc circles} include:
\begin{enumerate}
\item
  A Fortran 77 line that ended on one line with a $+$ symbol and started on the next line with
  another $+$ symbol (within the valid columns for standard Fortran 77) was apparently accepted
  by the {\sc gcc} family fortran compiler in 2008, but not now (2020). One of the $+$ symbols
  was removed.
\item
  A fitting algorithm {\sc gsl\_multifit\_covar} available in GNU Scientific Library ({\sc GSL})
  versions 1.x was obsoleted; it is no longer present in modern 2.x versions of {\sc GSL}. With
  the aim of reproducing the calculations as close as possible as to the way they were done
  at the time of the original project, {\sc GSL-1.10} was downloaded and compiled from source.
\end{enumerate}

\subsection{Running}

At the time of writing the original paper, I felt that the use of GNU
tools to provide a free-licensed package configurable and compilable
with {\tt ./configure \&\& make} and a detailed {\tt ./circles
  -{}-help} command would be sufficient to enable easy reproduction by
a scientifically competent user. For example, invokin the {\sc
  circles} help option shows both single-hyphen, one-character options
and their equivalent double-hyphen, long options, such as
\mbox{{\tt -i,  -{}-cmb\_file\_raw=FILE cmb fits file of input data}}.
\sloppy

\fussy
In addition, I also stored my own private notes of what I intended to
be most of the significant steps taken in carrying out the project.
However, in trying to reproduce these steps now, I see that my notes
are not as complete as they should be. For the purpose of
reproducibility, I have had to prepare a completely fresh {\sc bash}
script that verifies the sha512 checksums of input data files and
software packages that are downloaded from source rather than provided
within the security context of the host system. This is partly based
on more recent attempts at reproducibility in my own recent papers in
which {\sc git} commit hashes of the
software\supercite{Roukema17silvir} and a {\sc bash} script for
running the full software\supercite{RO19flatness} were provided.

This is an example of the problem of ``insider knowledge'' being
required for the reproducibility of a paper, where ``insider'' also
includes knowledge that may still be coded in the scientist's brain,
but not in written form.


\begin{enumerate}
\item
  TODO...
\end{enumerate}
