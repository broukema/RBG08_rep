

\abstract{\articleABSTRACT} %% seems to be missing in the template


\section{Introduction}

The aim in \emph{this} paper is to reproduce the results that use the
method described in Section 3.2 and the observational
analysis results described in Section 4.2, displayed in Figs.~3, 4, 5
and given numerically in Tables 2 and 3, where the numbers here
indicate those in RBG08.

\section{Method}

The first steps planned for trying to reproduce the original results
were to read the appropriate sections of RBG08.
\begin{enumerate}
\item
  Section 2.1\supercite{RBG08} states that the analysis method of Section 3.2 requires
  the three files at URLs listed in footnotes 1, 2, 3 on the same page. These files
  represent two versions of an all-sky map of the Universe mostly representing cosmic microwave background
  emission at 10{\hGpc} (comoving) from the Earth as observed by
  the Wilkinson Microwave Anisotropy Probe (WMAP)\supercite{WMAP5Hinshaw}, and
  the ``kp2'' mask to enable analysis that avoids the most contaminated regions of
  the sky. These files are downloaded.
\item
  Footnote 7\supercite{RBG08} indicates that {\sc circles-0.3.2.1}, to be found at
  the URL \url{http://cosmo.torun.pl/GPLdownload/dodec/}, provides the software
  for generating the figures and tables. This software is downloaded as
  \url{http://cosmo.torun.pl/GPLdownload/dodec/circles-0.3.2.1.tar.gz}.
\end{enumerate}

The next step was to develop a script on a {\sc git} repository server
that satisfies the requirements of the international scientific community,
specifically the International Science Council\supercite{ISCFreedoms}, by
not blocking access to scientists of any countries or territories.
As of 2020, several popular {\sc git} repository servers block access to
scientists and other citizens of several territories\supercite{Github2020}.


\begin{table}
  \begin{tabular}{ll}
    \hline
    ded213c1c4cfdfe2ef92f7155b27d58c & wmap\_ilc\_5yr\_v3.fits \\
    fbc8b2518fdddf0a1e7b5acde99a748e & wiener5yr\_map.fits \\
    5aa3267dc6d69bf8c5f0a3a893e23960 & wmap\_kp2\_r9\_mask\_3yr\_v2.fits \\
    afbd67d8120c11e949eb0c414c2775f5 & circles-0.3.2.1.tar.gz \\
    \hline
  \end{tabular}
  \caption{Checksums (md5sums) of the data and the main software
    source code files of RBG08, use for the present reproducibility
    test.\protect\label{t-md5sums}}
\end{table}

\section{Results}

The overall script intended to carry out the full sequence of
downloads, configuring of packages,
compiling of packages,
subdirectory user-level installation of packages,
setting up of calculation parameters, and
running the main code, was set up as a {\sc bash} script
{\tt reproduce\_RBG08.sh}.

The full package aiming to reproduce the figures and tables listed
above is provided at \url{https://codeberg.org/boud/0807.4260},
named after the ArXiv identity of RBG08.

\subsection{Downloading data and software source code}

\begin{enumerate}
\item
  The URL in footnotes 1, 2 and 3\supercite{RBG08} gave clickable
  links that were split into two and not correctly clickable. The user
  needs to cut/paste the two halves of each URL in order to obtain the
  three data files. The data files were downloaded with no apparent
  problem, with md5sums as indicated in Table~\ref{t-md5sums}.
\item
  The file {\sc circles-0.3.2.1.tar.gz} with the md5sum indicated in
  Table~\ref{t-md5sums} was downloaded. It was included in
  the main git repository in its original form. Subsequent changes
  are recorded in the git history at
  \url{https://codeberg.org/boud/0807.4260}.
\end{enumerate}

\subsection{Compiling/debugging}

Fixes needed in order to successfully compile {\sc circles} include:
\begin{enumerate}
\item
  A Fortran 77 line that ended on one line with a $+$ symbol and
  started on the next line with another $+$ symbol (within the valid
  columns for standard Fortran 77) was apparently accepted by the {\sc
    gcc} family fortran compiler in 2008, but not now (2020). One of
  the $+$ symbols was removed.
\item
  A fitting algorithm {\sc gsl\_multifit\_covar} available in GNU
  Scientific Library ({\sc GSL}) versions 1.x was obsoleted; it is no
  longer present in modern 2.x versions of {\sc GSL}. With the aim of
  reproducing the calculations as close as possible as to the way they
  were done at the time of the original project, {\sc GSL-1.10} was
  downloaded and compiled from source.
\item
  Using the modern {\sc gfortran} compiler options {\tt -fcheck=bounds
    -Wall} to highlight likely sources of bugs due to insufficiently
  standard coding led to many warnings.
  Checking of these warnings motivated many fixes that could either
  solve errors in running the main {\sc circles} package, or reduce
  the chance of unexpected code errors.
\item
  The autotools {\tt autoreconf} command was run in the main {\sc
    circles-0.3.2.1} directory and its subdirectories.
\item
  The {\sc cosmdist} package provided by default in a subdirectory of
  {\sc circles-0.3.2.1} was replaced by a
  download/configure/compile/install section of the main reproduction
  script, since {\sc cosmdist} is now available in an online
  git repository.
  some of
\item
  Memory allocation errors that occur for running {\sc circles}
  without previously definining environment variables for key
  values such as input filenames are present in {\sc
    circles-0.3.2.1}. These were most likely not noticed in RBG08
  because of the use of environment variables providing these
  values. Fixes to the front-end C file {\sc circles.c} were
  made with the attention of avoiding memory allocation errors,
  which are typically reported to the user as segmentation faults.
\end{enumerate}


\subsection{Running/debugging}

At the time of writing the original paper, I felt that the use of GNU
tools to provide a free-licensed package configurable and compilable
with {\tt ./configure \&\& make} and a detailed {\tt ./circles
  -{}-help} command would be sufficient to enable easy reproduction by
a scientifically competent user. For example, invokin the {\sc
  circles} help option shows both single-hyphen, one-character options
and their equivalent double-hyphen, long options, such as
\mbox{{\tt -i,  -{}-cmb\_file\_raw=FILE cmb fits file of input data}}.
\sloppy

\fussy
In addition, I also stored my own private notes of what I intended to
be most of the significant steps taken in carrying out the project.
However, in trying to reproduce these steps now, I see that my notes
are not as complete as they should be. For the purpose of
reproducibility, I have had to prepare a completely fresh {\sc bash}
script that verifies the sha512 checksums of input data files and
software packages that are downloaded from source rather than provided
within the security context of the host system. This is partly based
on more recent attempts at reproducibility in my own recent papers in
which {\sc git} commit hashes of the
software\supercite{Roukema17silvir} and a {\sc bash} script for
running the full software\supercite{RO19flatness} were provided.

This is an example of the problem of ``insider knowledge'' being
required for the reproducibility of a paper, where ``insider'' also
includes knowledge that may still be coded in the scientist's brain,
but not in written form.

A more fundamental problem in terms of coding and software
environment evolution is that this code uses a C front end and
a Fortran 77 backend, compiled together using {\sc autoconf}
tools, in a way that, to the best of my knowledge and that of
my co-authors, worked correctly in 2008.

The main files are {\tt circles\_f77.f}, named to emphasise its
expected obsolescence, 2023 lines long; 21 Fortran 77
source files with 11,616 lines of code in {\tt lib/}; and
3415 lines of Fortran 77 code in the auxiliary package
{\tt astromisc/lib/}. For modernisation of this code, a minimum
approach would be to convert the Fortran 77 to Fortran 2008,
in which case Fortran--C compatibility conventions are reasonably
well developed.

For the purposes of this reproducibility test, the required re-coding
effort would be more than is presently justified. While this is, as
far as I know, the only free-licensed code for identified-circles
matching for the purposes of cosmic topology analysis for which
peer-reviewed research has been published, the techniques developed in
codes that are not publicly available under either free or non-free
licences have developed considerably since 2008.  Moreover, the
research field is extremely high risk: an observational confirmation
of the spatial topology of the Universe would be a historically
important discovery, but whether or not this measurement is feasible
remains highly speculative. To serve as a basis for long-term
projects in this particular field, the software would best be
rewritten according to modern standards of C and/or Fortran and
benefitting from a decade of improvements in free-licensed geometrical
software libraries such as {\sc cgal} that could potentially be
useful.

Thus, this reproducibility attempt was terminated with
an untraced bug that is very likely to be memory related. The behaviour
of the bug (or bugs) is inconsistent, yielding errors such as
\mbox{{\tt Fortran runtime error: Substring out of bounds:}}
\mbox{{\tt upper bound (7) of 'bin\_format' exceeds string length (4)}} or
\mbox{{\tt Fortran runtime error: Substring out of bounds:}}
\mbox{{\tt upper bound (2) of 'indir' exceeds string length (-1158212918)}}.
\sloppy

\section{Conclusion}

\fussy
It is ironical that in the field of cosmic topology, not only are most
software packages only available privately, with unknown licences, but
the code that is explicitly public with peer-reviewed published
results has turned out to be less easy to reproduce than
expected. This is, unfortunately, consistent with typical reports on
science research paper reproducibility.  The specific bottleneck
suspected of leading to memory errors in this case was that the effort
required to update the Fortran 77 files at the heart of the code,
interfaced with a C front end, risked being too great to be justified
on any short time scale. While Fortran has remained actively used by
scientists since more than half a century ago and in its modern
standards continues to be used actively, and the original {\sc
  circles} package was prepared using the powerful GNU {\sc
  autotools}, a robust interface and standards for compiling C and
Fortran code together have only evolved quite recently.

While the results of the paper are not as reproducible as they
appeared to be, the requirement of the Ten Years Challenge
for the code to be placed in an online git repository,
which in this case is \url{https://codeberg.org/boud/0807.4260},
resulted in confirmation that the source code is
fully free-licensed, including all libraries and other
auxiliary software packages, and the input data files
remain publicly available online.
