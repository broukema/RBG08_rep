% DO NOT EDIT - automatically generated from metadata.yaml

\def \codeURL{https://bitbucket.org/broukema/0807.4260}
\def \codeDOI{}
\def \codeSWH{}
\def \dataURL{}
\def \dataDOI{}
\def \editorNAME{}
\def \editorORCID{}
\def \reviewerINAME{}
\def \reviewerIORCID{}
\def \reviewerIINAME{}
\def \reviewerIIORCID{}
\def \dateDRAFTED{}
\def \dateRECEIVED{}
\def \dateACCEPTED{}
\def \datePUBLISHED{}
\def \articleTITLE{Reproducibility of 'Poincaré dodecahedral space parameter estimates'}
\def \articleTYPE{Replication}
\def \articleDOMAIN{astronomy}
\def \articleBIBLIOGRAPHY{RBG08_rep.bib}
\def \articleYEAR{2020}
\def \reviewURL{}
\def \articleABSTRACT{The surface-of-last-scattering optimal cross-correlation method of finding a preferred orientation of the fundamental domain of the spatial section of the Universe, under the working hypothesis that the spatial section is a Poincaré dodecahedral space developed in earlier work, was published in 2008 (RBG08). The steps required to download the Wilkinson Microwave Anisotropy Probe (WMAP) five-year data and the software code used in RBG08 and reproduce the main results of that paper are examined here to see if the earlier work is reproducible.}
\def \replicationCITE{Roukema, B. F.; Buliński, Z.; Gaudin, N. E., 2008, Poincaré dodecahedral space parameter estimates, Astronomy and Astrophysics, 492, 657-673}
\def \replicationBIB{RBG08}
\def \replicationURL{https://ui.adsabs.harvard.edu/link_gateway/2008A\%26A...492..657R/PUB_PDF}
\def \replicationDOI{10.1051/0004-6361:200810685}
\def \contactNAME{Boudewijn F. Roukema}
\def \contactEMAIL{boud astro.uni.torun.pl}
\def \articleKEYWORDS{rescience c, fortran, cosmology}
\def \journalNAME{ReScience C}
\def \journalVOLUME{4}
\def \journalISSUE{1}
\def \articleNUMBER{}
\def \articleDOI{}
\def \authorsFULL{Boudewijn F. Roukema}
\def \authorsABBRV{B.F. Roukema}
\def \authorsSHORT{Roukema}
\title{\articleTITLE}
\date{}
\author[1,2,\orcid{0000-0002-3772-0250}]{Boudewijn F. Roukema}
\affil[1]{Institute of Astronomy, Faculty of Physics, Astronomy and Informatics, Nicolaus Copernicus University, Grudziadzka 5, 87-100 Toruń, Poland}
\affil[2]{Univ Lyon, Ens de Lyon, Univ Lyon1, CNRS, Centre de Recherche Astrophysique de Lyon UMR5574, F–69007, Lyon, France}
