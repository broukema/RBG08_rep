% DO NOT EDIT - automatically generated from metadata.yaml

\def \codeURL{https://codeberg.org/boud/0807.4260}
\def \codeDOI{}
\def \codeSWH{}
\def \dataURL{}
\def \dataDOI{}
\def \editorNAME{}
\def \editorORCID{}
\def \reviewerINAME{}
\def \reviewerIORCID{}
\def \reviewerIINAME{}
\def \reviewerIIORCID{}
\def \dateDRAFTED{}
\def \dateRECEIVED{}
\def \dateACCEPTED{}
\def \datePUBLISHED{}
\def \articleTITLE{Reproducibility of 'Poincaré dodecahedral space parameter estimates'}
\def \articleTYPE{Replication}
\def \articleDOMAIN{astronomy}
\def \articleBIBLIOGRAPHY{RBG08_rep.bib}
\def \articleYEAR{2020}
\def \reviewURL{}
\def \articleABSTRACT{The surface-of-last-scattering optimal cross-correlation method of finding a preferred orientation of the fundamental domain of the spatial section of the Universe, under the working hypothesis that the spatial section is a Poincaré dodecahedral space developed in earlier work, was published in 2008 (RBG08). The steps required to download the Wilkinson Microwave Anisotropy Probe (WMAP) five-year data and the software code used in RBG08 to reproduce that paper's main results were examined here to see how easy it is to reproduce them. It was found that the effort in updating the Fortran 77 code at the heart of the software, interfaced with a C front end, risked being too great to be justified on any short time scale. In that sense, the results of RBG08 are not as reproducible as they appeared to be, despite the source code being fully free-licensed and the input data files remaining publicly available. Cosmic topology papers by other groups are based on private software. The software and a script to reproduce the steps of this incomplete reproduction are combined in a new git repository named 0807.4260, using the ArXiv identity code of RGB08.}
\def \replicationCITE{Roukema, B. F.; Buliński, Z.; Gaudin, N. E., 2008, Poincaré dodecahedral space parameter estimates, Astronomy and Astrophysics, 492, 657-673}
\def \replicationBIB{RBG08}
\def \replicationURL{https://ui.adsabs.harvard.edu/link_gateway/2008A\%26A...492..657R/PUB_PDF}
\def \replicationDOI{10.1051/0004-6361:200810685}
\def \contactNAME{Boudewijn F. Roukema}
\def \contactEMAIL{boud astro.uni.torun.pl}
\def \articleKEYWORDS{rescience c, fortran, cosmology}
\def \journalNAME{ReScience C}
\def \journalVOLUME{4}
\def \journalISSUE{1}
\def \articleNUMBER{}
\def \articleDOI{}
\def \authorsFULL{Boudewijn F. Roukema}
\def \authorsABBRV{B.F. Roukema}
\def \authorsSHORT{Roukema}
\title{\articleTITLE}
\date{}
\author[1,2,\orcid{0000-0002-3772-0250}]{Boudewijn F. Roukema}
\affil[1]{Institute of Astronomy, Faculty of Physics, Astronomy and Informatics, Nicolaus Copernicus University, Grudziadzka 5, 87-100 Toruń, Poland}
\affil[2]{Univ Lyon, Ens de Lyon, Univ Lyon1, CNRS, Centre de Recherche Astrophysique de Lyon UMR5574, F–69007, Lyon, France}
