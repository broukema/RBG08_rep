% DO NOT EDIT - automatically generated from metadata.yaml

\def \codeURL{https://codeberg.org/boud/0807.4260}
\def \codeDOI{}
\def \codeSWH{}
\def \dataURL{}
\def \dataDOI{}
\def \editorNAME{}
\def \editorORCID{}
\def \reviewerINAME{}
\def \reviewerIORCID{}
\def \reviewerIINAME{}
\def \reviewerIIORCID{}
\def \dateDRAFTED{}
\def \dateRECEIVED{}
\def \dateACCEPTED{}
\def \datePUBLISHED{}
\def \articleTITLE{Reproducibility of 'Poincaré dodecahedral space parameter estimates'}
\def \articleTYPE{Replication}
\def \articleDOMAIN{astronomy}
\def \articleBIBLIOGRAPHY{RBG08_rep.bib}
\def \articleYEAR{2020}
\def \reviewURL{}
\def \articleABSTRACT{Is a scientific research paper based on (i) public, online observational data files and (ii) providing free-licensed software for reproducing its results easy to reproduce by the same author a decade later? This paper attempts to reproduce a cosmic topology observational paper published in 2008 and satisfying both criteria (i) and (ii). The reproduction steps are defined formally in a free-licensed git repository package ``0807.4260'' and qualitatively in the current paper. It was found that the effort in upgrading the Fortran 77 code at the heart of the software, interfaced with a C front end, and originally compiled with g77, in the content of the contemporary gfortran compiler, risked being too great to be justified on any short time scale. In this sense, the results of RBG08 are not as reproducible as they appeared to be, despite both (i) data availability and (ii) free-licensing and public availability of the software. The software and a script to reproduce the steps of this incomplete reproduction are combined in a new git repository named 0807.4260, following the ArXiv identity code of RBG08.}
\def \replicationCITE{Roukema, B. F.; Buliński, Z.; Gaudin, N. E., 2008, Poincaré dodecahedral space parameter estimates, Astronomy and Astrophysics, 492, 657-673}
\def \replicationBIB{RBG08}
\def \replicationURL{https://ui.adsabs.harvard.edu/link_gateway/2008A\%26A...492..657R/PUB_PDF}
\def \replicationDOI{10.1051/0004-6361:200810685}
\def \contactNAME{Boudewijn F. Roukema}
\def \contactEMAIL{boud astro.uni.torun.pl}
\def \articleKEYWORDS{rescience c, fortran, cosmology}
\def \journalNAME{ReScience C}
\def \journalVOLUME{4}
\def \journalISSUE{1}
\def \articleNUMBER{}
\def \articleDOI{}
\def \authorsFULL{Boudewijn F. Roukema}
\def \authorsABBRV{B.F. Roukema}
\def \authorsSHORT{Roukema}
\title{\articleTITLE}
\date{}
\author[1,2,\orcid{0000-0002-3772-0250}]{Boudewijn F. Roukema}
\affil[1]{Institute of Astronomy, Faculty of Physics, Astronomy and Informatics, Nicolaus Copernicus University, Grudziadzka 5, 87-100 Toruń, Poland}
\affil[2]{Univ Lyon, Ens de Lyon, Univ Lyon1, CNRS, Centre de Recherche Astrophysique de Lyon UMR5574, F–69007, Lyon, France}
